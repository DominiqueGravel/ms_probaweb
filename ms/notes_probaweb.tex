

% ANALYSIS OF CONTINENTAL WIDE VARIATION IN NETWORK STRUCTURE
	% Dataset description

	% Description of the parametrization
		% Data format
		
		% Fit models one pair of species
		% Compare the different maps
		% Reconstruct the interaction probability 
		% Evaluate the likelihood and AIC for the whole network

	% Interpretation 
		% Model comparison 
		% Drivers of spatial variation in network structure
		% Implications for the evaluation of the metaweb


% DISCUSSION

% TO DO FOR THE ANALYSIS

% - Format the data

% - Fit models for the interactions
%	P(Lij | Xi, Xj)
% 	P(Lij | Xi, Xj, E)

% - Fit models for the distribution
% 	P(Xi,Xj)
% 	P(Xi,Xj | E )
% 	P(Xi)P(Xj)
% 	P(Xi | E)P(Xj | E)

% - Fit models for the joint distribution
% 	P(Lij)
% 	P(Lij|E)

% - Compute the likelihood for each model

% - Random network generator
% 	- species (doable only for independent distributions)
% 	- links

% FIGURES

% - Illustration of network sampling. A) The metaweb of host-parasitoid interactions across Europe and B) a local network structure. 

% - Conceptual representation of the integrated niche

% - Mapping the drivers of spatial variation of network structure at the landscape scale
% 	A) host species i (observed & predicted)
% 	B) parasitoid species j (observed & prediced)
% 	C) co-occurrence of host & parasitoids (observed & predicted)
% 	D) interactions (observed & predicted by the integrated niche model)

% - Evaluation of the metaweb quality (range of the 95% condifence interval)

% - Network structure over environmental gradients
% 	A) Map of connectance with the integrated network
% 	B) Connectance as a function of temperature (with only species distribution - with only the interaction - integrated model)
% 	C) Map of nestedness
% 	D) Nestedness as a function of temperature (with only species distribution - with only the interaction - integrated model) 

% TABLES

% - Model comparison (model description, nb parameters, likelihood, AIC)





We developed the \texttt{probaweb} package for R (REF) to fit
alternative formulations of the metaweb and the co-occurrence matrix
along an environmental gradient and run it to re-interpret the data of
Tylianaks (2007). The package provides a general interface facilitating
the development of different species and link distribution models. It is
also built to facilitate the interaction with the\texttt{mangal} database
of ecological interactions (REF). The first step consists of fitting a
probabilistic model from the observation of a pairwise interaction (binary)
and the environment (could be categorical or continuous) from the subset of
the data where the two species are co-occurring. In other words, it fits
the equation $P(L_{ijy}|X_{iy},X{jy},E_y)$ to the data where $X_{iy} = 1$
and $X_{iy} = 1$. Logistic regression was used and is currently programmed,
but alternative models could be used as well. The second steps consists of
fitting a probabilistic model for co-occurrence over the whole dataset,
$P(X_{iy},X{jy}|E_y)$, independently of the observation of an interaction. The
two probabilities are then multiplied to obtain the probability of observing
an interaction (Eq. 2). We used this probability to compute the likelihood of
each observation ($\zeta(\theta|D) = P(L_{ijy},X_{iy},X_{jy})$ if $L_{ijy}=1$
and $\zeta(\theta|D) = 1 - P(L_{ijy},X_{iy},X_{jy})$ otherwise). We then
after compare the models by their AIC.







We considered the gradient of habitat modification as a ordered categorical
variable and compared XX models (results are summarized at Table 2). Not
surprisingly the best model takes into account the effect of the environment on
both the metaweb and co-occurrence. What is most interesting are the comparisons
to the best model. First, we find that using a constant metaweb has a dramatic
impact on the fit of the model to the data (the AIC drops from X for model 1 to
X for model 2), indicating a strong effect of the environment on pairwise
interactions. Secondly, we find that the deterministic metaweb is the worst
model (model 3, AIC = ). This result indicate that the traditional approach to
consider that species interact as soon as they co-occur is definitely wrong.
Thirdly, we also find that using a constant co-occurrence does have a
significant impact on the model (the AIC drops to X, model 4), indicating there
is a non-random change in community composition with habitat modification. Taken
together, these two results better explain why network structured changed with
habitat modification, even though here we only used binary information about the
network structure. Another interesting result is that considering a neutral
co-occurrence did not impact much the fit of the model. The AIC drops to XX
with model 6, indicating that considering independent SDMs yields similar
networks over this environmental gradient. This means that for this particular
dataset, ecological interactions does not have a strong impact on species
distribution since; a strong dependence of parasitoids to the host for instance
would have a occurrence probability higher than expected by chance, while a
repulsion would have had the opposite.

An important output of this analysis is a more explicit representation of the
uncertainty in the evaluation of the metaweb. We find that among the XX pairs of
host and parasitoids, XX did not co-occur. There were therefore many forbidden
links based on co-occurrence. These might never occur in reality, but we do not
know without doing extra experiments. Therefore, any analysis of the structure
of the metaweb would be inappropriate without filling those gaps. In addition to
specific experiments, the gaps could be filled with a trait-based approach,
using phylogenies or with a null hypothesis (e.g. the interaction probability is
equal to connectance computed on the observed interactions).

It is also possible to obtain for each pairwise interaction an estimate of
the uncertainty. Not surprisingly, the confidence interval is usually high
for the estimation of a probability with a small sample size. The standard
error on the evaluation of the interaction probability is provided along
with the metaweb at Fig. 3. It reveals that the uncertainty is high for
most interactions, even if 48 networks were sampled. Such an approach could
be used to detect which pairwise interaction requires additional sampling
in order to reduce the uncertainty to a manageable level.